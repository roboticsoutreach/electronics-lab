\documentclass{article}

\usepackage{hyperref}
\usepackage{siunitx}

\title{Electronics Lab}

\begin{document}

\maketitle

% TODO: introduction

\section{Setup}

On your desk you should find:

\begin{itemize}
\item a breadboard (a white plastic board with a grid of holes)
\item pliers
\item wire strippers
% etc.
\end{itemize}

If some of these items aren't present or if you're having trouble identifying
them, please ask for assistance from one of the organisers.

\subsection{Hookup wire}

We will use single-strand hookup wire to connect up our circuit. Some
electronics kits provide pre-cut wires with connectors attached to each end;
however being able to cut and strip wire off the reel is a good skill to learn.

Using the wire cutters, cut roughly \SI{30}{\centi\metre} of black and red wire
from the reels. Conventially, red wire is used for anything directly connected
to the positive power supply and black wire is used for anything directly
connected to the negative power supply (ground).

Also cut about \SI{60}{\centi\metre} of any other colour. This wire will be used
for any other connections in the circuit. Feel free to use different colours to
represent different sections of the circuit (e.g. blue for input circuitry and
green for output circuitry), or anything else that may help you keep track of
the layout of your circuit better.

Practise cutting a few short lengths of wire and stripping the insulation off
the ends. You should leave between \SI{5}{\milli\metre} and
\SI{10}{\milli\metre} of bare copper at the end; too little and the wire won't
stay securely in the breadboard hole, too much and you risk exposed copper
making contact nearby component leads or other wires and causing a short
circuit. If you are unfamiliar with the process of stripping wires, ask an
organiser to demonstrate the procedure.

\subsection{Breadboard}

In this lab, we will build our circuits on a breadboard, which is a useful
platform for mounting and connecting components that doesn't require soldering.
Some of you may have used a breadboard before; if not, there is an excellent
guide to getting started with breadboards available on SparkFun's website at
\url{https://learn.sparkfun.com/tutorials/how-to-use-a-breadboard}.

The breadboards should have two power supply rows at the top, one labelled with
a red line (usually used for the positive power supply) and one with a blue or
black line (for the negative power supply). There should also be another two
rows like these at the bottom of the board. Using a piece of (preferably red)
hookup wire, connect the two red rows together. Do the same for the two
black/blue rows.

\subsection{Battery clip}

For this lab we'll use a standard PP3 (9 volt) battery as our power supply.
Grab the battery clip and strip the ends of the leads if necessary. Plug the
red (positive) lead into one of the positive power supply rows of the
breadboard, and the black (negative) lead into one of the negative power supply
rows.

Don't connect the clip to the battery just yet; we'll do that when we're ready
to test our circuit. It's good practice to keep the power to your circuit
disconnected except when necessary.

\section{Wiring up an LED}

% is this too easy? presumably if they've signed up for a paid electronics
% course, they already have a significant interest in electronics.

Right way round.

Current limit.

\section{Controlling an LED with a button}

\section{Controlling an LED's brightness with a variable resistor}

\section{Controlling an LED from the Joint IO board}

\section{Reading a button from the Joint IO board}

\section{Ultrasound sensor?}

\section{Using a servo}

\end{document}


% Questions:
%   Are bench PSUs available?
%   Are participants working in pairs or on their own?
%   Should hookup wires be provided, or should participants cut and strip their
%    own wire?
%   Do we need to buy pliers, wire strippers etc. or will ECS loan us these?

% TODO:
%   "If you need assistance, talk to blueshirts" etc.
%   
