\documentclass{article}

\title{Electronics Lab}

\begin{document}

\maketitle

\section{Setup}

On your desk you should find:

\begin{itemize}
\item a breadboard (a white plastic board with many rectangular holes)
\item pliers
% etc.
\end{itemize}

Inside the breadboard, the topmost row of holes are all connected to each other.
Usually, this is used to provide easy access to the power supply for components
in the main working area of the breadboard. Plug the positive (red) lead of the
battery holder into any hole on this top row.

Likewise, the bottom row of holes are also connected to each other, and are used
for the negative power supply (ground). Connect the negative (black) lead of the
battery holder into the bottom row.

The rest of the breadboard is the main working area, where you can build and
test circuits. Each group of 5 holes arranged in a column are connected
together.

Put batteries in battery holder.

Cut and strip hookup wire.

\section{Wiring up an LED}

Right way round.

Current limit.

\section{Controlling an LED with a button}

\section{Controlling an LED's brightness with a variable resistor}

\section{Controlling an LED from the Joint IO board}

\section{Reading a button from the Joint IO board}

\section{Ultrasound sensor?}

\section{Using a servo}

\end{document}


% Questions:
%   Are bench PSUs available?
%   Are participants working in pairs or on their own?
%   Should hookup wires be provided, or should participants cut and strip their
%    own wire?
%   

% TODO:
%   "If you need assistance, talk to blueshirts" etc.
%   
